%%% Introduction %%%

Legged robotic systems have long employed motion controllers based on limit cycle oscillators and, more recently, 
Central Pattern Generators (CPGs)  for the purpose of generating bio-mimetic gaits 
\cite{Matsuoka1985,Collins1993,Endo2004,Righetti2006,Ijspeert2008,Matos2010,Ajallooeian2013,Park2014,Fukuoka2015}. 
Since these motion control methods are, in essence, open-loop motion planners they do not guarantee stability and cannot 
be applied directly without further consideration of the legged system's dynamics. Moreover, proper implementation of 
the aforementioned methods of locomotion requires the use of auxiliary control mechanisms which can adapt the system 
to disturbances and changes in environmental conditions.

Namely, developments in CPG-based gait controller have led to the incorporation ``reflexive" feedback
mechanisms aimed at correcting foot-placement and trunk posture during gaiting \cite{Fukuoka2003,Endo2004}. These methods 
are meant to further extend to similarity of legged systems to their biological counterparts, and hinge on fixed-point control methods
covered in detail in \cite{Wieber2015}. However, few gait controllers of this nature, however, utilize an on-line learning mechanism 
to handle the task of achieving system stability, namely angular trunk stability.

Disturbance rejection from the angular body states of is of practical importance for larger legged systems with many sensors 
rigidly fixed to their trunk sections. By extension, the trunk of a legged system could be used to pitch and roll sensor units,
like camera's and laser distance sensors, as will be done with our \emph{BlueFoot} quadruped system, shown in \MissingFig. Performing
such a task during gaiting certainly demands proper control stability control of angular trunk states. Without compensation 
vision sensor measurements will likely be disturbed by vibrations caused by instantaneous changes in force distribution, namely, 
when feet make and break contact with the ground. Additionally, certain dynamic gaits during which two or fewer legs are in 
contact with the ground at any given time cause the system to enter under-actuated state where the trunk is free to rock 
about the planted feet, as shown in Figure \ref{fig::tipping_robot}.  These gaits however, are advantageous because they 
typically allow quadruped systems to attain higher land speeds than with static creep-gaits \MissingRef. 
	\begin{figure}[t!]
		\centering
		\SetImage{\ImageWidthRatio}{tipping_robot.png}
		\caption{Quadruped Tipping About Planted Feet During A Trot Gait}
		\label{fig::tipping_robot}
	\end{figure}
To address the problem of trunk-state disturbance rejection during CPG-generated, non-static gaits, this paper will
present a trunk-stabilization routine that utilizes a Nonlinear Autoregressive eXogenous (NARX)-model Nueral Network 
to learn and predict the quadruped system  during gaiting and predict periodic disturbances.

The use of a NARX-model Neural Network is particularly important feature of the control method being presented, 
as it has been shown that network models have an affinity for approximating nonlinear difference systems and make 
chaotic time-series predictions \cite{Tsungnan1996,ChenBillings1990,Hihi1996,Billings2013}. This provides
a natural fit to a problem such as this, where the dynamics at are both periodic and of a high enough complexity
where an approximation method is certainly warranted.

This paper will first examine the general-form dynamics for a legged system, and a first-order discrete-time counterpart. 
Next, the nuero-compensator mechanism will be outlined, and the NARX-Network training regimen will be described with respect
to the system dynamics. This will also include a section about how the compensator output is applied to correct joint reference 
signals for when the compensator is applied to a legged system implemented with a decentralized joint control architecture. 
Lastly, results showing the effectiveness of the compensator, as it is applied to a quadruped platform (the \emph{BlueFoot} 
Quadruped) executing a CPG-drive trot gait, will be presented. Results will highlight the robustness of this approach,
even when applied to gaiting over a surface with unperceived 